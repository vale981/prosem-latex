\documentclass[presentation,t,aspectratio=169]{beamer}
\usepackage{siunitx}
\usepackage[utf8]{inputenc}
\usepackage[font={scriptsize,it}]{caption}
\usepackage{hyperref}
\usepackage[T1]{fontenc}
\sisetup{prespace}
\usetheme{Antibes}
\setbeamertemplate{itemize items}[default]
\setbeamertemplate{enumerate items}[default]

\usepackage{stackengine}
\newcommand{\figcite}[3]{
  \def\stackalignment{c}
  \stackunder{\includegraphics[width=#1]{#2}}{\tiny Source: \url{#3}}
}
\newcommand{\btVFill}{\vskip0pt plus 1filll}


% \logo{\includegraphics[height=0.5cm]{logo.png}} 
\newcommand{\inlineMovie}[5][autostart&loop]
{
    \href{run:#2?#1}{\figcite{#4}{#3}{#5}}
}

\setbeamertemplate{footline}[frame number]


\AtBeginSection[]
{
   \begin{frame}
       \tableofcontents[currentsection]
   \end{frame}
}


\title{Detection of Exoplanets}
\subtitle{Part 1 - Exoplanet Overview}
\author{Valentin Boettcher}
\beamertemplatenavigationsymbolsempty

\begin{document}
\begin{frame}
  \titlepage
\end{frame}

{  
  \usebackgroundtemplate{
  \includegraphics[width=\paperwidth]{material/peg_b.jpg}}
  \begin{frame}[plain]
  \end{frame}
}

\begin{frame}
  \frametitle{Outline}
  \tableofcontents[pausesections]
\end{frame}

\section{The definition of an Exoplanet}
\begin{frame}
  \frametitle{What is an Exoplanet?}
  \begin{block}{Origins of the Term}
    Exoplanet / Extrasolar planet from Greek 'Exo' (Latinized)
    - ``Outside''.
  \end{block}
  \pause
  \begin{definition}
    There is none!
  \end{definition}
  \begin{block}{Some distinction from Stars}
    \begin{itemize}
    \item Fusion: does not fuse hydrogen (or any other element)
      \\ $\implies$ Mass: moderately massive
    \item Position: in orbit around a star
    \item Formation: formed from accretion disk around a star
    \end{itemize}    
  \end{block}
\end{frame}
  \begin{frame}
    \frametitle{A Working-Defintion}
    \begin{block}{IAU 2003 Recommendation}
      Planets $=$ Objects below 13 Jupiter masses.
    \end{block}
    \pause
    Another Proposal based on the solar system definition by Jean-Luc Margot:
    \begin{itemize}
    \item based on metric for the clearing of the orbit
      \begin{itemize}
      \item based on mass of start and planet, orbital period
      \end{itemize}
    \item classifies 99\% of the known planets
    \end{itemize}
  \end{frame}

\section{Nomenclature and Units}
\begin{frame}
  \begin{block}{Nomenclature}    
     Name/Designation of Star +  a Letter ('b', 'c', ...)
   \end{block}
   \pause
  \begin{example}
    51 Pegasi b - first confirmed exoplanet around a normal star
  \end{example}
  \pause
  \begin{block}{Units}
    \begin{itemize}
    \item Degrees ($1\degree$), Arcminutes ($1\arcmin$), Arcsecond ($1\arcsec$) - Measure for seperation on an imaginary sphere
      \begin{itemize}
      \item $\SI{1}{\arcmin}= \frac{1}{60} \degree$,  $\SI{1}{\arcsec}= \frac{1}{60} \arcmin$
      \end{itemize}
    \item $\mathrm{AU}$ Astronomical Unit - mean distance Earth-Sun $= \SI{149597870700}{\km}$
    \item $\lightyear$ Lightyear - distance light travels in a year $= \SI{9.4607e15}{\meter}$
    \end{itemize}
  \end{block}
\end{frame}

% \subsection{Some Orbital Parameters}
% \begin{frame}
%   \frametitle{Some Orbital Parameters}
%   \begin{columns}[t]
%     \column{.7\textwidth}
%       \begin{itemize}
%       \item<1-> Period $P\;[\mathrm{days}]$
%       \item<2-> Semi Major Axis $a\;[\mathrm{au}]$
%       \item<3-> Eccentricity $\epsilon$
%       % \item<4->inclination $i\;[\degree]$ in respect to the reference plane
%       \end{itemize}
%     \column{.3\textwidth}
%     \begin{block}{}<2->
%       \includegraphics[width=\textwidth]{material/el.png}
%     \end{block}
%   %   \begin{block}{}<4->
%   %     \includegraphics[width=\textwidth]{material/inc.png}
%   %   \end{block}
%  \end{columns}
% \end{frame}

\section{Challenges in Exoplanet Detection}
\begin{frame}
  The problem with planets:
  \begin{itemize}
  \item usually small compared to star
    \pause
  \item close to the star
  \end{itemize}
  \pause
  \begin{example}
    Looking at the Solar System and 51 Pegasi ($d=\SI{50.9}{\lightyear}$):
    \begin{itemize}
    % \item Angular Separation of Eath: $\SI{0.06}{\arcsec}$
    \item Angular Separation of 51 Pegasi b as seen from Earth: $\SI{0.003}{\arcsec}$
    \end{itemize}
    \pause
    Out of league for most Telescopes.
  \end{example}
  \pause
  \begin{itemize}
  \item not lumnius themselves (in the visible band) \\
    $\implies$ \alert{very (very!) faint}
  \end{itemize}
  \pause
  $\longrightarrow$ only about 20 directly imaged planets
\end{frame}

\subsection{Direct Imaging}
\begin{frame}
  \frametitle{Direct Imaging}
  \pause
  \begin{enumerate}
  \item masking the star \pause
  \item taking a lot of images (in infrared) \pause
  \item stacking, interferometry and clever computer processing \pause
  \end{enumerate}

  \end{frame}
  % \begin{frame}
  %   \frametitle{Least Massive: Formalhaut b (2 Jup. Masses)}
  %     \begin{figure}
  %       \centering
  %             \includegraphics[width=0.8\textwidth]{material/formb.jpg}
  %     \caption{http://spacetelescope.org/images/html/heic0821a.html}
  %     \end{figure}
  % \end{frame}
    \begin{frame}
    \frametitle{Coldest: Gliese 504 b ($\SI{240}{\celsius}$)}
      \begin{figure}
        \centering
        \figcite{0.45\textwidth}{material/glies.jpg}{http://www.nasa.gov/content/goddard/astronomers-image-lowest-mass-exoplanet-around-a-sun-like-star/index.html}
      \end{figure}
  \end{frame}
    \begin{frame}
    \frametitle{HR 8799 has 4 Planets}
      \begin{figure}
        \centering
        \inlineMovie{material/keck_exo.avi}{material/keck_ex_p.png}{.75\textheight}{https://www.manyworlds.space/index.php/tag/hr-8799/}
      \end{figure}
  \end{frame}

\section{Historic Overview}
% \subsection{Antique}
% \begin{frame}
%   \frametitle{Antique}
%   \begin{itemize}
%   \item universe as a nebula \pause $\implies$ matter cant stay evenly
%     distributed forever
%   \end{itemize}

%   \pause
%   \begin{quote}
%     ``Many bodies of all sorts and shapes move from the infinite into a
%     great void; they come together there and produce a single whirl,
%     in which, colliding with one another and revolving in all manner
%     of ways, they begin to separate like to like.'' \hfill - \tiny{\textit{LEUCIPPUS, 480-420 B.C.}}
%   \end{quote}
% \pause
%   \begin{quote}
%     ``There \alert{cannot be more worlds} than one world.'' \\ \hfill - \tiny{\textit{ARSITOTLE, 384–322 B.C.}}
%   \end{quote}
% \end{frame}

\begin{frame}
  \begin{columns}[t]
    \column{0.7\textwidth}
  \begin{itemize}
  \item<1-> \small{\textit{KOPERNIKUS}} (1543) of course supported plurality
  \item<2-> \small{\textit{GIORDANO BRUNO}} (1548-1600) postulates spacial infinity
  \item<3-> \small{\textit{NEWTON}} (1548-1600) speculates about other solar systems
  \item<4-> \small{\textit{OTTO STRUVE}} (1952) proposes methods of exoplanet detection
  \end{itemize}
  \only<5>{$\longrightarrow$ In the following years: A lot of false-positives.}
  \column{0.3\textwidth}
  \only<1>{\figcite{\textwidth}{material/kop.jpg}{Icones, p. 36}}
  \only<2>{\figcite{\textwidth}{material/brun.jpg}{Neue Bibliothec}}
   \only<3>{\figcite{\textwidth}{material/newt.jpg}{National Portrait Gallery: NPG 2881}}
  \only<4->{\figcite{\textwidth}{material/struve.jpg}{US Post}}
\end{columns}
\end{frame}

\subsection{First Planets}
\begin{frame}
  \frametitle{First Planets}
  \begin{columns}
    \column{0.5\textwidth}
    \begin{itemize}
    \item<1-> 1992 - first radio pulsar timing planet
    \end{itemize}
    \action<2->{
    \begin{block}{1995 - First planet around 'real' star}
      \begin{itemize}
      \item discovered by Mayor, Queloz of University Geneva
      \item 4 day orbit, much closer than Mercury (Hot Jupiter), 1.9 Jupiter radii 
      \end{itemize}
    \end{block}}
  \only<1>{\column{0.5\textwidth}\figcite{\textwidth}{material/puls.jpg}{https://photojournal.jpl.nasa.gov/catalog/PIA08042}}
      \only<2>{\column{0.5\textwidth}\figcite{\textwidth}{material/peg_b.jpg}{https://www.eso.org/public/russia/images/eso1517a/}}
  \end{columns}
\end{frame}

\subsection{State-of-the-Art}
\begin{frame}
  \frametitle{Today's State-of-the-Art}
  \only<1>{
    \begin{center}
      \figcite{.9\textwidth}{material/exoplanets_1.png}{http://exoplanets.org/}      
    \end{center}
  }
  \only<2>{
    \begin{center}
      \figcite{.9\textwidth}{material/exoplanets.png}{http://exoplanets.org/}      
    \end{center}
  }
\end{frame}

\begin{frame}
  \frametitle{KEPLER Space Telescope}
  \begin{columns}[T]
    \column{0.5\textwidth}
    \begin{itemize}
    \item space observatory by \textit{NASA} to find planets
    \item 2344 confirmed, over 2000 unconfirmed candidates
    \end{itemize}
    \column{0.5\textwidth}
    \figcite{\textwidth}{material/kepler.jpg}{https://www.nasa.gov/mission_pages/kepler/}
  \end{columns}
\end{frame}

\subsection{Future}
\begin{frame}
  \frametitle{GAIA Space Telescope}
  \begin{columns}[T]
    \column{0.5\textwidth}
    \begin{itemize}
    \item creates 3D Star Map since 2013
    \item goal: find 10.000 - 50.000 Exoplanets (0 found yet)
    \end{itemize}
    \column{0.5\textwidth}
    % \includegraphics[witdh=0.2\textwidth]{material/kepler.jpg}
    \figcite{\textwidth}{material/gaia.jpg}{http://www.jwst.nasa.gov/images_artist13532.html}
  \end{columns}
\end{frame}
\begin{frame}
  \frametitle{JWST James Web Space Telescope}
  \begin{columns}[T]
    \column{0.5\textwidth}
    \begin{itemize}
    \item to-be successor of Hubble for infrared spectrum
    \item delayed to 2021 (originally launch 2013)
    \end{itemize}
    \column{0.5\textwidth}
    \figcite{\textwidth}{material/jwst.jpg}{http://www.jwst.nasa.gov/images_artist13532.html}
  \end{columns}
\end{frame}


{  
  \begin{frame}[plain]
    \begin{center}
      \includegraphics[height=0.9\paperheight]{material/progress.jpg}      
    \end{center}
  \end{frame}
}

\section{The Diversity of Exoplanets}
\begin{frame}
  \frametitle{Some Facts}
  \begin{columns}
    \column{.5\textwidth}
      \begin{itemize}
  \item exoplanets orbit all kinds of stars
  \item mass range: 0.02 to 5780 Earth Masses
  \item orbital period range: \SIrange{0.4}{3.2e5}{\day}
  \item semi major axis: \SIrange{0.00585}{113}{AU}
  \item furthest away: \SI{21190}{\lightyear}
  \end{itemize}
  \column{.5\textwidth}
  \figcite{\textwidth}{material/mass_sep.png}{http://exoplanets.org/plots}
  \end{columns}
\end{frame}

\subsection{Exoplanet Extremes}
\begin{frame}
  \frametitle{Hot Jupiters}
  \begin{columns}
    \column{0.5\textwidth}
    \begin{itemize}
    \item within a few Jupiter masses
    \item close to star (\SIrange{0.015}{0.05}{AU}) $\implies$ hot
    \item easy to detect by radial velocity method, astrometry
    \end{itemize}
    \column{0.5\textwidth}
    \figcite{\textwidth}{material/hj.jpg}{http://hubblesite.org/newscenter/archive/releases/2008}
  \end{columns}
\end{frame}

\begin{frame}
  \begin{columns}
    \frametitle{Planets in Habitable Zone}
    \column{0.5\textwidth}
    \begin{itemize}
    \item<1-> habitable zone: allows liquid water, tollerable radiation levels
    \item<2-> nearest: Proxima Centauri b $\SI{4.2}{\lightyear}$
    \item<3-> extrapolations of KEPLER $\rightarrow$ more than 40 billion
      in milkyway
    \end{itemize}
    \column{0.5\textwidth}
    % \includegraphics[witdh=0.2\textwidth]{material/kepler.jpg}
    \action<2->{\figcite{\textwidth}{material/prox_b.jpg}{https://www.eso.org/public/images/eso1629a/}}
  \end{columns}
\end{frame}

{  
  \begin{frame}[plain]
    \begin{center}
      \includegraphics[height=0.9\paperheight]{material/habit.jpg}      
    \end{center}
  \end{frame}
}

\begin{frame}
  \frametitle{Systems with many Planets}
  \begin{columns}
    \column{0.5\textwidth}
    \begin{itemize}
    \item<1-> Trappist-1 has 7 planets, some in habtialble zone
    \item<2-> hard to detect, but supposed to be prevalent
    \item<3-> exoplanet.eu lists 637 multiple planet systems
    \end{itemize}
    \column{0.5\textwidth}
    \figcite{\textwidth}{material/trap.jpg}{http://photojournal.jpl.nasa.gov/figures/PIA22093_fig1.jpg}
  \end{columns}
\end{frame}
  
{
  \usebackgroundtemplate{
    \includegraphics[width=\paperwidth]{material/trapbig.jpg}}
  \begin{frame}[plain]
    \bigskip
    \hfill  \color{white} Thanks for your attention!

    \btVFill
   
    \hfill\tiny\color{white}\url{https://www.spacetelescope.org/images/heic1713a/}\bigskip
  \end{frame}
}

\section*{Resources}
\begin{frame}
  \frametitle{(Re)Sources}
  \begin{itemize}
  \item \scriptsize \url{http://exoplanets.org/} big, accurate database \normalsize
  \item \scriptsize \url{http://exoplanet.eu/} another database \normalsize
  \item \scriptsize \url{https://exoplanets.nasa.gov/} 3D visualizations! \normalsize
  \item \tiny{``PROPOSAL FOR A PROJECT OF HIGH-PRECISION STELLAR RADIAL VELOCITY WORK''} by Otto Struve \normalsize
  \item \tiny{``A QUANTITATIVE CRITERION FOR DEFINING PLANETS''}  by Jean-Luc Margot \normalsize
  \item \tiny{``A COMBINED VERY LARGE TELESCOPE AND GEMINI STUDY OF THE ATMOSPHERE
      OF THE DIRECTLY IMAGED PLANET, BETA PICTORIS b''} by Thayne Currie et al. \normalsize
  \item \tiny{``TEMPERATE EARTH-SIZED PLANETS TRANSITING A NEARBY
ULTRACOOL DWARF STAR''} by
    Michaël Gillon  et al. \normalsize
  \item \tiny{``NO LARGE POPULATION OF UNBOUND OR WIDE-ORBIT
      JUPITER-MASS PLANETS''} by Przemek Mróz et al. (Rogue Planets, not included in the talk) \normalsize 

  \end{itemize}
\end{frame}
\end{document}
